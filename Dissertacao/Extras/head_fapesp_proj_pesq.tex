% ------------------------------------------------------------------------
% Projeto de pesquisa - FAPESP
%
% Modelo criado por: Rodrigo Romano
% "Rodrigo A. Romano" <rromano@maua.br>
%
% Alteraćoes: Rafael Corsi
% "Rafael Corsi" <corsiferrao@gmail.com>
% 
% ------------------------------------------------------------------------
\documentclass[12pt,a4paper]{article}
\usepackage[a4paper,left=3.5cm,right=1.5cm,top=3cm,bottom=2.5cm]{geometry}
\usepackage[utf8]{inputenc}       %Permite acentuação
\usepackage[T1]{fontenc}
\usepackage[english,brazil,brazilian]{babel}
\usepackage{graphicx}        %Permite inserção de figuras
\usepackage[centertags]{amsmath}
\usepackage{amssymb,amsfonts}
\usepackage{schemabloc}
\usepackage{pst-node, pstricks}
\usepackage{ncccomma}
\usepackage{indentfirst}
\usepackage{array}					% Hifenizaçao em tabelas
\usepackage{multirow}
%\usepackage[numbers,comma]{natbib}
\usepackage{url, hyperref}		% Site
\usepackage{lmodern}

%\usepackage{natbib}

%Matem figura na secção !
%\usepackage[section]{placeins}

% sub figuras
\usepackage{subfigure}
%ajuste do tamanho entre figuras
\setlength\subfigcapmargin{0.8em}

%cores diversas 
\usepackage{color}


% Fuzz -------------------------------------------------------------------
\hfuzz1pt % Don't bother to report over-full boxes if over-edge is < 2pt

% Numeração das notas de rodapé ------------------------------------------
\renewcommand{\thefootnote}{\fnsymbol{footnote}}

% Different font in captions ---------------------------------------------
\newcommand{\captionfonts}{\small}

\makeatletter  % Allow the use of @ in command names
\long\def\@makecaption#1#2{%
  \vskip\abovecaptionskip
  \sbox\@tempboxa{{\captionfonts #1: #2}}%
  \ifdim \wd\@tempboxa >\hsize
    {\captionfonts #1: #2\par}
  \else
    \hbox to\hsize{\hfil\box\@tempboxa\hfil}%
  \fi
  \vskip\belowcaptionskip}
\makeatother   % Cancel the effect of \makeatletter

% Espaçamento ------------------------------------------------------------
\setlength{\parindent}{30pt} \setlength{\parskip}{6pt}
\newlength{\defbaselineskip}
\setlength{\defbaselineskip}{\baselineskip}

\newcommand{\setlinespacing}[1]{\setlength{\baselineskip}{#1 \defbaselineskip}}
% ---
\newcommand{\PreserveBackslash}[1]{\let\temp=\\#1\let\\=\temp}
\let\PBS=\PreserveBackslash %
% ------------------------------------------------------------------------
\def\baselinestretch{1}
\setlinespacing{1.5}
% ------------------------------------------------------------------------

%-----------Alteraćoes Rafael C ----------------------------------------
% Use the microtype package for better typography
% http://www.khirevich.com/latex/microtype/
%
% activate={true,nocompatibility} - activate protrusion and expansion
% final - enable microtype; use "draft" to disable
% tracking=true, kerning=true, spacing=true - activate these techniques
% factor=1100 - add 10% to the protrusion amount (default is 1000)
% stretch=10, shrink=10 - reduce stretchability/shrinkability (default is 20/20)
\usepackage[activate={true,nocompatibility},final,tracking=true,kerning=true,spacing=true,factor=1100,stretch=10,shrink=10]{microtype} 

% Possibilita que caracteres saiam da margem p/ melhor caber o texto
\SetProtrusion{encoding={*},family={bch},series={*},size={6,7}}
             {1={ ,750},2={ ,500},3={ ,500},4={ ,500},5={ ,500},
               6={ ,500},7={ ,600},8={ ,500},9={ ,500},0={ ,500}}

\microtypecontext{spacing=nonfrench}
               
% Todo notes
\usepackage[disable]{todonotes}
%\usepackage[colorinlistoftodos]{todonotes}  
\setlength{\marginparwidth}{3cm}
\reversemarginpar             
\makeatletter\let\chapter\@undefined\makeatother

%changes
\usepackage{doc}

%You may also want to reset the footnote counter for each page. People are not familiar with the symbols beyond the double dagger, and more than (at most) 16 footnotes in a single chapter will result in an error message. To reset the counter per page, either use the footmisc option perpage or the following code:
\usepackage{perpage}
\MakePerPage{footnote}

% historico de mudanças
\usepackage[]{vhistory}

%\setcounter{secnumdepth}{5}
%\setcounter{tocdepth}{5}
