\pagestyle{empty}
\cleardoublepage
\pagestyle{fancy}

\chapter{Modelagem Eletromagnética}

Passos da modelagem:

\begin{enumerate}[a.]
	\item modelagem eletromagnética da parte passiva $\rightarrow$ Força passiva dependente de x,y,z
	\item modelagem eletromagnética da parte ativa  $\rightarrow$ Força ativa de pedente de x,y,z,i
	\item fundir modelos
	\item modelagem da dinâmica e acoplamentos 
\end{enumerate}

%Dimensões do mancal (Fig. \ref{Fig:Modelagem:Dimensoes})

\begin{figure}[!ht]
	\centering
	\def\svgwidth{\columnwidth}
	\includesvg{modelo:dim}
		\caption{Dimensões do mancal}
		\label{Fig:Modelagem:Dimensoes}
\end{figure}

\section{Circuito passivo}

A parte passiva do mancal magnético pode ser descrita como o circuito da Fig. \ref{Fig:Modelagem:circuito:passivo}, onde um imã permanente gera um fluxo magnético que estabiliza o eixo axial (passivo) e possibilita que o eixo radial seja estabilizado com um menor gasto energético.

%Os passos para obtenção do modelo são:
%
%\begin{enumerate}[a.]
%	\item modelo sem saturação no ferro
%	\item modelo com saturação no ferro
%	\item modelo mais complexo com vazamento de campo
%\end{enumerate}

%\begin{figure}[!ht]
%	\centering
%	\def\svgwidth{\columnwidth}
%	\includesvg{modelo:circuito:passivo}
%	\caption{Circuito magnético passivo}
%	\label{Fig:Modelagem:circuito:passivo}
%\end{figure}

Sendo:
\begin{itemize}
	\item $\mathcal{F}_m$ : Forca Eletromotriz gerado pelo imã permanente;
	\item $\phi_m$ : Fluxo magnético imã permanente;
	\item $\phi_m$ : Fluxo magnético gap;
	\item $\mathcal{R}_m$ : Relutância devido ao imã permanente;
	\item $\mathcal{R}_{feeX}$ : Relutância devido ao ferro do estator externo (T: Topo, B: Baixo);
	\item $\mathcal{R}_{geX}$ : Relutância devido ao gap externo (T: Topo, B: Baixo);
	\item $\mathcal{R}_{geX}$ : Relutância devido ao ferro do rotor.
\end{itemize}

O mancal foi seccionado em oito partes, sendo que cada parte possui uma força de atração relacionada. Para o cálculo dessas forças, utilizou-se uma zona de integração de $45^{\circ}$ com o centro alinhando com a direção da força. A Fig. \ref{Fig:Modelagem:circuito:passivo:forca} ilustra as forças de atração agindo no rotor e mostra em verde a zona de integração utilizada para o cálculo das forças.

\begin{figure}[!ht]
	\centering
	\def\svgwidth{0.8\columnwidth}
	\includesvg{modelo:circuito:passivo:forcas}
	\caption{Distribuição de forças no rotor devido a parte passiva do mancal magnético}
	\label{Fig:Modelagem:circuito:passivo:forca}
\end{figure}

\subsection{Campo magnético no entreferro}

Com a premissa de que todas as linhas de campo passem pelo gap e que o \textit{fringe} possa ser desprezado, e que os ferros do estator externo estão saturados \todo[noline]{Interagir com a simulacão}, obtemos das leis de Marxell para campos quase estáticos um modelo que relaciona a força de atração magnética pela distância do rotor ao estator esterno (comprimento do gap) \todo[noline]{Verificar termo: Quasi-Static e Fringe} \todo[noline,color=magenta]{Ref. Livro Edward E. Furlani}:


\todo[inline]{o campo no ferro do estator externo e’ saturado mesmo, estava confundindo com o campo no estrator interno.
no estator externo a saturação é necessária, pois maximiza o B absoluto e minimiza o efeito diferencial, o que aumenta a rigidez axial (passiva) sem aumentar muito a rigidez radial.}


\begin{eqnarray}
	\phi_m = \phi_f = \phi_g \label{eq:p:fluxo}
\end{eqnarray}

\begin{align}
	\oint{ \textbf{H } \cdot d\textbf{l}} &= \cancelto{0}{I_t} \notag\\ 
	\rightarrowtail	H_m l_m + 2 H_f l_f + 2 H_g l_g &= 0 \label{eq:p:linhas}
\end{align}

Podemos derivar da Eq. \ref{eq:p:fluxo}:

\begin{align}
	B_m S_m &= B_f S_f \label{eq:p:BmSm} \\ 
	B_f S_f &= B_g S_g \label{eq:p:BfSf} \\ 
	B_g S_g &= B_m S_m \label{eq:p:BgSg}
\end{align}

A relação $B_m/H_m$ para imãs de terra rara como Samário-Cobalto ou Neodímio é praticamente linear podendo ser aproximada para:


\begin{align}
	B_m &= B_r - \frac{B_r}{H_c}H_m \label{eq:p:Bm}
\end{align}


Como estamos trabalhando na zona de saturação: $H > H_{sat}$ (Fig. \ref{Fig:Modelagem:Curva:BH}) adotamos que a zona de operação da curva B-H é linear e possui a seguinte equação :

\begin{align}
	B_f &= B_{fs} + \mu_s (H_f - H_{fs}) \label{eq:p:Bf}
\end{align}

\begin{figure}[!ht]
	\centering
	\def\svgwidth{0.7\columnwidth}
	\includesvg{modelo:curva:bh}
		\caption{Curva de saturação de um material magnético}
		\label{Fig:Modelagem:Curva:BH}
\end{figure}

Substituindo \eqref{eq:p:Bm} e \eqref{eq:p:Bf} na equação \eqref{eq:p:BmSm}


\[	[B_r - \frac{B_r}{H_c}H_m]S_m = \left[ B_{fs} + \mu_s (H_f - H_{fs}) \right] S_f \]

Isolando $H_m$ em função de $H_f$ :

\begin{align}
	H_m &= \left[ -\left[B_{fs} + \mu_s (H_f - H_{fs}) \frac{S_f}{S_m}\right] - Br \right] \frac{H_c}{H_m} \\
	%
	&=  - H_f \frac{\mu_s S_f}{S_m}\frac{H_c}{H_m} + \left( \mu_s H_{fs} \frac{S_f}{S_m} + B_r - B_{fs}\right) \frac{H_c}{H_m}
	\label{eq:p:Hm}
\end{align}

Obtemos da Eq. \eqref{eq:p:BfSf}:

\begin{align*}
	B_f S_f &= B_g S_g \notag \\
%	B_f S_f &= \mu_0 H_g S_g \notag \\
	[B_{fs} + \mu_s (H_f - H_{fs})] S_f &= \mu_0 H_g S_g \notag \\
\end{align*}

Isolando $H_g$ em função de $H_f$ :

\begin{equation}
	H_g = \frac{B_{fs} - \mu_s S_f H_{fs}}{\mu_0 S_g} + H_f \frac{\mu_s S_f}{\mu_0 S_g} \label{eq:p:Hg}
\end{equation}

Substituindo na Eq. \eqref{eq:p:linhas}  os valores de $H_g$, $H_m$ obtemos o campo magnético no ferro:

\[
	 \left(  - H_f \frac{\mu_s S_f}{S_m}\frac{H_c}{H_m} + \left( \mu_s H_{fs} \frac{S_f}{S_m} + B_r - B_{fs}\right) \frac{H_c}{H_m}\right) l_m + 2 H_f l_f + 2 \left( \frac{B_{fs} - \mu_s S_f H_{fs}}{\mu_0 S_g} + H_f \frac{\mu_s S_f}{\mu_0 S_g} \right) l_g = 0
\]

\begin{multline}
	H_f \left[ -\frac{\mu_s S_f }{S_m}\frac{H_c}{H_m} l_m + 2 l_f + 2 \frac{\mu_s S_f}{\mu_0 S_g} l_g\right] = \\
				 -\left( \mu_s H_{fs} \frac{S_f}{S_m} + B_r - B_{fs}\right) \frac{H_c}{H_m} l_m - 2 \left( \frac{B_{fs} - \mu_s S_f H_{fs}}{\mu_0 S_g} \right) l_g
\end{multline}


Definido as variáveis auxiliares

\begin{align}
	C_1 &= -\frac{\mu_s S_f }{S_m}\frac{H_c}{H_m} l_m + 2 l_f \\
	C_2 &= 2 \frac{\mu_s S_f}{\mu_0} \\
	C_3 &= -\left( \mu_s H_{fs} \frac{S_f}{S_m} + B_r - B_{fs}\right)  \frac{l_m H_c}{H_m}\\
	C_4 &= - 2 \left( \frac{B_{fs} - \mu_s S_f H_{fs}}{\mu_0} \right) 
\end{align}


%\begin{multline}
%	H_f = \left[-\left( \mu_s H_{fs} \frac{S_f}{S_m} + B_r - B_{fs}\right) \frac{H_c}{H_m} l_m - 2 \left( \frac{B_{fs} - \mu_s S_f H_{fs}}{\mu_0 S_g} \right) l_g \right] \; \\
%				 \left[-\frac{\mu_s S_f }{S_m}\frac{H_c}{H_m} l_m + 2 l_f + 2 \frac{\mu_s S_f}{\mu_0 S_g} l_g\right]^{-1}
%\end{multline}

Obtemos o valor do vetor campo magnético $H_f$ :

\begin{eqnarray}
	H_f = \frac{C_1 + C_2 \, \nicefrac{l_g}{S_g}}{C_3 + C_4 \, \nicefrac{l_g}{S_g}}
\end{eqnarray}

Sendo $B_f = B_{fs} + u_{fs} (H_f-H_{fs})$ podemos então, calcular o valor do campo magnético no entreferro \eqref{eq:p:Hg}:

\begin{equation}
	B_g = \frac{B_f \; S_f}{S_m}
\end{equation}

\subsection{Decomposição do campo magnético B em X e Z} \label{SubSec:CampoX/Y}

O campo magnético acumulado no entreferro pode ser decomposto em componentes $B_x$ e $B_z$ que dependem do deslocamento do rotor em $\Delta_x$ e $\Delta_z$, esse deslocamento implica também em um aumento no comprimento do gap: $l_g$, A Fig. \ref{Fig:modelo:passivo:DxDz} ilustra o deslocamento. Tal modelo não leva en consideração o \textit{tilt} do rotor, que implicaria em relutâncias diferentes para a parte superior e inferior do rotor.

	\begin{figure}[!ht]
		\centering
		\def\svgwidth{0.7\columnwidth}
		\includesvg{modelo:passivo:DxDy}
			\caption{Deslocamento em X e Y}
			\label{Fig:modelo:passivo:DxDz}
	\end{figure}

 Os campos podem então ser derivados:
 
 \begin{align}
 	\theta &= tg^{-1}(\frac{\Delta_z}{\Delta_x}) \\
 	l_g &= \sqrt{\Delta_x^2 + \Delta_z^2} \\
 	B_{gx} &= B \, cos(\theta) \\
 	B_{gz} &= B \, sin(\theta) 
 \end{align}

\subsection{Parâmetros Físicos}

As áreas e comprimentos médios são obtidos da seguinte maneira (com uma zona de integração de $45^{\circ}$), com referência a Fig. \ref{Fig:Modelagem:Dimensoes} e a tabela em Anexo \ref{Tabela:Modelagem:Dimensoes}: \todo[noline,color=magenta]{Gerar tabela de dim}

\begin{align}
	S_m &= w_m     	\frac{2 \pi (r_{eei} + w_{fee} - w_m)}{8} \\
	S_f &= h_{fee} 	\frac{2 \pi r_{eei}}{8} \\
	S_g(l_g) &= h_{fee}	\frac{2 \pi (r_{reei} - \nicefrac{l_g}{2})}{8} \alpha_g \\
	l_m &= h_m \\
	l_f &= w_{fee}
\end{align}

\todo[inline]{Explicar melhor o que é espraiamento e como encontrar-lo}

O Termo $ \alpha_g $ é o fator de espraiamento do campo magnético no entreferro do mancal  que é devido a dispersão do campo magnético, ou seja a área em que o entreferro acumula campo magnético é sempre maior ($\alpha_g > 1)$ do que a área calculada.  

\subsection{Força por trabalho virtual}

\todo[inline]{Dedicar um tempo para explicar o conceito}

A força magnética é então calculado por :

\begin{align}
	\vec{F} &= -\frac{ \vec{B}_{g}^2 \; S_g}{\mu_0} \label{eq:passivo:Fx}
\end{align}

\subsection{Equações de força em X}


Desejamos obter as resultantes das forças projetadas nos eixos radiais (x e y), para obtermos um modelo mais preciso das forças, o mancal foi divido em oito partes distintas sendo que cada parte possui um componente de campo magnético diferente das outras partes. Para então obtermos o valor das forças radiais, precisamos calcular todas as forças e então decompor-las nos eixos. Similar com a força em Y :

\begin{align}
	F_x &= F_{L} + F_{NLx} + F_{SLx} - F_{O} - F_{NOx} - F_{SOx} \ref{eq:p:F:resultante:x}
\end{align}

\subsection{Equações de força em Y}

A força perpendicular ao plano de rotação é a composição de todas as forças exercidas em z (não considerando tilt) :

\begin{equation}
	  		F_z = \frac{S_{g}}{\mu_0} 	\sum_{i=N}^{NO} B_{giz}^2
\end{equation}


\newpage
\section{Circuito Ativo}

Para o circuito ativo, utilizaremos dois diferentes modelos, um para o ferro não saturado e outro para valores de corrente que saturem o ferro do núcleo. Essa abordagem é tomada para minimizar erros numéricos.


\subsection{Modelo sem saturação}

Pela lei de Ampere:

\begin{align}
	H_{fn} l_{fn} + H_{gn} l_{gn} + H_{fa} l_{fa} + H_{ga} l_{ga} + H_{ra} l_{ra}= \mathcal{F} \label{eq:at:loop1} \\ 
	H_{fn} l_{fn} + H_{gn} l_{gn} + H_{fb} l_{fb} + H_{gb} l_{gb} + H_{rb} l_{rb}= \mathcal{F}  \label{eq:at:loop2}
\end{align}

Da conservação do fluxo magnético:

\begin{eqnarray}
	B_{gn} S_{gn} =  B_{ga} S_{ga} + B_{gb} S_{gb} \notag \\
	\rightarrow H_{gn} S_{gn} =  H_{ga} S_{ga} + H_{gb} S_{gb}  \label{eq:at:bgnsgn}
\end{eqnarray}

núcleo

\begin{align}
	B_{gn} S_{gn} = B_{fn} S_{fn} \notag \\
		H_{gn} S_{gn} \mu_0 = H_{fn} S_{fn} \mu \notag \\
		H_{fn} = \frac{H_{gn} S_{gn} \mu_0}{S_{fn} \mu} \label{eq:at:hfn}
\end{align}

loop I

\begin{align}
	B_{ga} S_{ga} = B_{fa} S_{fa} \notag \\
		H_{ga} S_{ga} \mu_0 = H_{fa} S_{fa} \mu \notag \\
		H_{fa} = \frac{H_{ga} S_{ga} \mu_0}{S_{fa} \mu} \label{eq:at:hfa}
\end{align}

\begin{align}
	B_{ga} S_{ga} = B_{ra} S_{ra} \notag \\
		H_{ga} S_{ga} \mu_0 = H_{ra} S_{ra} \mu \notag \\
		H_{ra} = \frac{H_{ga} S_{ga} \mu_0}{S_{ra} \mu} \label{eq:at:hra}
\end{align}

 loop II

\begin{align}
	B_{gb} S_{gb} = B_{fb} S_{fb} \notag \\
		H_{gb} S_{gb} \mu_0 = H_{fb} S_{fb} \mu \notag \\
		H_{fb} = \frac{H_{gb} S_{gb} \mu_0}{S_{fb} \mu} \label{eq:at:hfb}
\end{align}

\begin{align}
	B_{gb} S_{gb} = B_{rb} S_{rb} \notag \\
		H_{gb} S_{gb} \mu_0 = H_{rb} S_{rb} \mu \notag \\
		H_{rb} = \frac{H_{ga} S_{gb} \mu_0}{S_{rb} \mu} \label{eq:at:hrb}
\end{align}

Substituindo as equações \eqref{eq:at:hfa} e \eqref{eq:at:hra} em \eqref{eq:at:loop1} :
\begin{align}
	\frac{H_{gn} S_{gn} \mu_0}{S_{fn} \mu} l_{fn} + 
	H_{gn} l_{gn} + 
	\frac{H_{ga} S_{ga} \mu_0}{S_{fa} \mu} l_{fa} + 
	H_{ga} l_{ga} + 
	\frac{H_{ga} S_{ga} \mu_0}{S_{ra} \mu} l_{ra} 
	= 
	\mathcal{F} \notag
	\\  % % 
	H_{gn} \left[
				\frac{S_{gn} l_{fn} \mu_0}{S_{fn} \mu}  + l_{gn}
		   \right] +
    H_{ga} \left[ 
		    	\frac{S_{ga} l_{fa} \mu_0}{S_{fa} \mu}  + 
		    	l_{ga} + 
		    	\frac{S_{ga} l_{ra} \mu_0}{S_{ra} \mu} 
   		   \right]
 	= 
 	\mathcal{F} \notag
 	\\ %
 	H_{ga} 
 	=  
 	\left(
	 	\mathcal{F} -
	 	H_{gn} \left[
	 				\frac{S_{gn} l_{fn} \mu_0}{S_{fn} \mu}  + l_{gn}
	 			\right] 
 	\right)
 	\left[ 
 		   	\frac{S_{ga} l_{fa} \mu_0}{S_{fa} \mu}  +  
 		   	\frac{S_{ga} l_{ra} \mu_0}{S_{ra} \mu}  +
 		   	l_{ga} 
 	\right]^{-1} \label{eq:at:H_ga(H_gn)}
\end{align}

Assim como na equação anterior, podemos encontrar a dependência de $H_{gb}(H_{gn})$ com \eqref{eq:at:hfb} e \eqref{eq:at:hrb} em \eqref{eq:at:loop2}:

\begin{align}
	H_{gb} 
 	=  
 	\left(
	 	\mathcal{F} -
	 	H_{gn} \left[
	 				\frac{S_{gn} l_{fn} \mu_0}{S_{fn} \mu}  + l_{gn}
	 			\right] 
 	\right)
 	\left[ 
 		   	\frac{S_{gb} l_{fb} \mu_0}{S_{fb} \mu}  + 
 		   	\frac{S_{gb} l_{rb} \mu_0}{S_{rb} \mu} +
 		   	l_{gb}
 	\right]^{-1} \label{eq:at:H_gb(H_gn)}
\end{align}

Pela Eq. \eqref{eq:at:bgnsgn} :

\begin{multline*}
		H_{gn} S_{gn} =  
		\left(
			 	\mathcal{F} -
			 	H_{gn} \left[
			 				\frac{S_{gn} l_{fn} \mu_0}{S_{fn} \mu}  + l_{gn}
			 			\right] 
		 	\right)
		 	\left[ 
		 		   	\frac{S_{ga} \mu_0}{S_{fa} \mu} l_{fa} +  
		 		   	\frac{S_{ga} \mu_0}{S_{ra} \mu} l_{ra} +
		 		   	l_{ga}
		 	\right]^{-1}  S_{ga} 
		 	+ \\
			\left(
			 	\mathcal{F} -
			 	H_{gn} \left[
			 				\frac{S_{gn} l_{fn} \mu_0}{S_{fn} \mu}  + l_{gn}
			 			\right] 
		 	\right)
		 	\left[ 
		 		   	\frac{S_{gb} l_{fb} \mu_0}{S_{fb} \mu}  +  
		 		   	\frac{S_{gb} l_{rb} \mu_0}{S_{rb} \mu}  +
		 		   	l_{gb}
		 	\right]^{-1} S_{gb} 
\end{multline*}

\begin{multline}
		H_{gn} S_{gn} =  
		\left(
			 	\mathcal{F} -
			 	H_{gn} \left[
			 				\frac{S_{gn} l_{fn} \mu_0}{S_{fn} \mu}  + l_{gn}
			 			\right] 
		 \right)
		 (
			\left[ 
				   	\frac{S_{ga} \mu_0}{S_{fa} \mu} l_{fa} +  
				   	\frac{S_{ga} \mu_0}{S_{ra} \mu} l_{ra} +
				   	l_{ga}
			\right]^{-1}  S_{ga} 
				 	+ \\
			\left[ 
			 		\frac{S_{gb} l_{fb} \mu_0}{S_{fb} \mu}  +  
			 		\frac{S_{gb} l_{rb} \mu_0}{S_{rb} \mu}  +
			 		l_{gb}
			\right]^{-1} S_{gb} 	\label{eq:at:H_gnS_gn}		 	
		 )
\end{multline}

Definindo as variáveis auxiliares:

\begin{align}
	C_1(S_{gn}, l_{gn}) &= 	\frac{S_{gn} \, l_{fn} \, \mu_0}{S_{fn} \mu}  + l_{gn} \\
	C_2(S_{ga}, l_{ga}) &= 	\left[ 
							   	\frac{S_{ga} \mu_0}{S_{fa} \mu} l_{fa} +  
							   	\frac{S_{ga} \mu_0}{S_{ra} \mu} l_{ra} +
							   	l_{ga}
							\right]^{-1}  S_{ga} \\
	C_3(S_{gb}, l_{gb}) &= 	\left[ 
						 		\frac{S_{gb} l_{fb} \mu_0}{S_{fb} \mu}  +  
						 		\frac{S_{gb} l_{rb} \mu_0}{S_{rb} \mu}  +
						 		l_{gb}
							\right]^{-1} S_{gb} 	
\end{align}

Obtemos a Eq. \ref{eq:at:H_gnS_gn} simplificada:

\begin{equation}
	H_{gn} S_{gn} = \left(
							\mathcal{F} - H_{gn}\, C_1
					\right)
					\left(
						C_2 + C_3					
					\right)
\end{equation}

Que implica em:

\begin{equation}
	H_{gn}  = \frac{\mathcal{F} \, (C_2 + C_3)}{S_{gn} + C_1 \, (C_2 + C_3)}
\end{equation}

Pelas equações: \eqref{eq:at:H_ga(H_gn)} e \eqref{eq:at:H_gb(H_gn)}:

\begin{align}
	H_{ga} = (\mathcal{F} - H_{gn} \, C_1) \frac{C_2 }{S_{ga}} \\
	H_{gb} = (\mathcal{F} - H_{gn} \, C_1)\frac{C_3 }{S_{gb}}
\end{align}

Podemos então deduzir o componente campo magnético dos gaps:

\begin{align}
	B_{gn} &= \mu_0 H_{gn} \\
	B_{ga} &= \mu_0 H_{ga} \\
	B_{gb} &= \mu_0 H_{gb} 
\end{align}

\subsection{Força por trabalho virtual}

A força magnética resultante é calculada através das resultantes de cada bobina:

\begin{align}
	\vec{F} = \vec{F}_{gn} + \vec{F}_{gn} + \vec{F}_{gn} \label{eq:ativo:F:resultante}
\end{align}

Onde a força resultante em cada bobina é calcula por:

\begin{align}
	\vec{F_{g}} = \frac{\vec{B}_g^2 \; S_g}{\mu_0} 
\end{align}




%\subsection{Modelo com saturação}
%
%Pela lei de Ampere:
%
%\begin{align}
%	H_{fn} l_{fn} + H_{gn} l_{gn} + H_{fa} l_{fa} + H_{ga} l_{ga} + H_{ra} l_{ra}= \mathcal{F} \label{eq:at:loop1:s} \\ 
%	H_{fn} l_{fn} + H_{gn} l_{gn} + H_{fb} l_{fb} + H_{gb} l_{gb} + H_{rb} l_{rb}= \mathcal{F}  \label{eq:at:loop2:s}
%\end{align}
%
%Da conservação do fluxo magnético:
%
%\begin{eqnarray}
%	B_{gn} S_{gn} =  B_{ga} S_{ga} + B_{gb} S_{gb} \notag \\
%	\rightarrow H_{gn} S_{gn} =  H_{ga} S_{ga} + H_{gb} S_{gb}  \label{eq:at:bgnsgn:s}
%\end{eqnarray}
%
%
%Porém agora consideramos que o núcleo está saturado e que a equação que relaciona B e H é:
%
%\begin{align}
%	B_f &= B_{fs} + \mu_s (H_f - H_{fs}) \label{eq:a:Bf}
%\end{align}
%
%portanto 
%
%\begin{align}
%	B_{gn} S_{gn} = B_{fn} S_{fn} \notag \\
%		(B_{fs} + \mu_s (H_f - H_{fs})) S_{gn} \mu_0 = H_{fn} S_{fn} \mu \notag \\
%		H_{fn} = \frac{(B_{fs} + \mu_s (H_f - H_{fs})) \,  S_{gn} \,  \mu_0}{S_{fn} \mu} \label{eq:at:hfn:S}
%\end{align}
%
%loop I
%
%\begin{align}
%	B_{ga} S_{ga} = B_{fa} S_{fa} \notag \\
%		H_{ga} S_{ga} \mu_0 = H_{fa} S_{fa} \mu \notag \\
%		H_{fa} = \frac{H_{ga} S_{ga} \mu_0}{S_{fa} \mu} \label{eq:at:hfa:s}
%\end{align}
%
%\begin{align}
%	B_{ga} S_{ga} = B_{ra} S_{ra} \notag \\
%		H_{ga} S_{ga} \mu_0 = H_{ra} S_{ra} \mu \notag \\
%		H_{ra} = \frac{H_{ga} S_{ga} \mu_0}{S_{ra} \mu} \label{eq:at:hra:s}
%\end{align}
%
% loop II
%
%\begin{align}
%	B_{gb} S_{gb} = B_{fb} S_{fb} \notag \\
%		H_{gb} S_{gb} \mu_0 = H_{fb} S_{fb} \mu \notag \\
%		H_{fb} = \frac{H_{gb} S_{gb} \mu_0}{S_{fb} \mu} \label{eq:at:hfb:s}
%\end{align}
%



\newpage
\section{Simulações}

 Foi utilizado como ferramenta de modelagem o Software de elementos finitos e multi física \textit{Comsol} \todo{isso deve estar descrito na metodologia}. Nas simulações foram utilizados a curva de histerese do Aço 1020 (Curva em anexo) \todo{Colocar tabela B-H em anexo}. O solido foi criado com uma revolução de $45^\circ$ simulando uma secção das oito na qual o modelo foi dividido.  A Fig. \ref{Fig:Simulacao:Passivo:Mesh} ilustra a malha utilizada na execução das simulações com um número aproximado de 19000 elementos.
 
  
 	\begin{figure}[!ht]
 		\centering
 		\includegraphics[width=0.5 \columnwidth,angle=0]{Figs/Simulacoes/Passivo/3D:Mesh=1,2.png}
 			\caption{Modelo Comsol do circuito passivo \\ Malha utilizada nos cálculos}
 			\label{Fig:Simulacao:Passivo:Mesh}
 	\end{figure}
 
 \todo[inline]{falar mais sobre o mesh e o método de elementos finitos}

\section{Circuito passivo}

 O módulo do campo magnético pode ser visualizado para dois casos distintos na Fig. \ref{Fig:Simulacao:Passivo:2D:B}. Nessa simulação, podemos verificar que o ferro está saturado (B=1.6T), que o modelo não possui uma quantidade significativa de linhas de campo magnético que não atravessam o entreferro e que a área do entreferro ($S_g$) possui um pequeno espraiamento aumento assim sua área resultando em uma ligeira diminuição no cálculo da força. Verificamos também que o ponto de operação do imã sofre uma pequena variação passando de 1.05T para 1.06T.

	\begin{figure}[!ht]
		\centering
		\caption*{Campo magnético (T)}
		\subfloat[t][$\Delta_x = 0.6mm$ e $\Delta_z = 0$]
		{
			\includegraphics[width=0.45\columnwidth]{Figs/Simulacoes/Passivo/2D:B:dx=0,6.png}
		} \label{Fig:Simulacao:Passivo:2D:B:dx=0.6}
		\subfloat[b][$\Delta_x = 1.2mm$ e $\Delta_z = 0$]
		{
			\includegraphics[width=0.45\columnwidth]{Figs/Simulacoes/Passivo/2D:B:dx=1,2.png}
		}	
		\caption{Modulo do campo magnético do modelo no Comsol do circuito passivo}\label{Fig:Simulacao:Passivo:2D:B:dx=1.2}
		\label{Fig:Simulacao:Passivo:2D:B}
	\end{figure}

% O gráfico da força magnética pela translação do rotor ilustrado na Fig. \ref{Fig:Modelagem:Curva:passivo:fxd} é somente de um oitavo da revolução do mancal e, a curva possui um valor de singularidade quando o rotor se aproxima da ferro do estator externo ($l_g = 0$). O gráfico mostra o valor da da força em x, utilizou-se a Eq. \eqref{eq:passivo:Fx} sem deslocamento em Z ($\Delta_z = 0$), o valor calculado é confrontado com o simulado por elementos finitos.
% 
%\begin{figure}[!ht]
%	\centering
%	\caption*{Força magnética (N) x Deslocamento em X (mm)}
%	\includegraphics[width=1 \columnwidth,angle=0]{Figs/Simulacoes/Passivo/modelo:circuito:passivo:fXd:calculado:comsol.pdf}
%		\caption{Comparativo do cálculo da força magnética pela Eq. \eqref{eq:passivo:Fx} e pelo modelo no Comsol com $\Delta_z = 0$}
%		\label{Fig:Modelagem:Curva:passivo:fxd}
%\end{figure}

% Podemos na figura verificar a não linearidade da força com a distância, essa propriedade é minimizada quando analisamos a força de atração de forma diferencial como ilustrado na Fig. \ref{Fig:Modelagem:Curva:passivo:frxd}

A força magnética de atração do rotor é ilustrado na Fig. \ref{Fig:Modelagem:Curva:passivo:frxd}, a curva foi levantada implementado as Eq. \eqref{eq:p:F:resultante:x} para o translação do rotor em apenas um eixo. Foram utilizado os valores nominais do protótipo. Verificamos que o modelo apresenta uma curva linear em termos de força de atração por deslocamento, o que era desejado, já que implica em uma simplicidade no modelo e por consequência na malha de controle.

\begin{figure}[!ht]
	\centering
	\caption*{Força magnética (N) x Deslocamento em X (mm)}
	\includegraphics[width=1 \columnwidth,angle=0]{Figs/Simulacoes/Passivo/modelo:circuito:passivo:fXd:calculado:comsol:dx}
		\caption{Linearização da força quando trabalhado em modo diferencial}
		\label{Fig:Modelagem:Curva:passivo:frxd}
\end{figure}


A Fig. \ref{Fig:Modelagem:Curva:passivo:dy:linhas} ilustra o resultado da simulação através de elementos finitos onde o rotor é transladado verticalmente de 1.2mm em ambas. Verificamos que as linhas de campo apresentam uma grande deformação se comparamos as linhas de campo do rotor alinhado com o estator externo ($\Delta_z = 0$), essas deformações apresentam um erro numérico no calculo da força já que supusemos na Subsec. \ref{SubSec:CampoX/Y} que as linhas de campo podem ser decompostas em x e z e que essa decomposição é diretamente relacionada com o deslocamento do rotor ($\theta$). 

\begin{figure}[!ht]
	\centering
	%\caption*{Força magnética (N) x Deslocamento x (mm)}
	\subfloat[t][$\Delta_x = 1.2mm$ e $\Delta_z = 0mm$]
	{
		\includegraphics[width=0.45 \columnwidth,angle=0]{Figs/Simulacoes/Passivo/2D:B:dy=0:linhas.png}	
	}	\label{Fig:Modelagem:Curva:passivo:dy:linhas:0}
	\subfloat[t][$\Delta_x = 1.2mm$ e $\Delta_z = 1.2$]
	{
		\includegraphics[width=0.45 \columnwidth,angle=0]{Figs/Simulacoes/Passivo/2D:B:dx=1,2:linhas.png}
	}	\label{Fig:Modelagem:Curva:passivo:dy:linhas:1,2}
	\caption{Linhas de campo magnético para deslocamentos na vertical}
	\label{Fig:Modelagem:Curva:passivo:dy:linhas}
\end{figure}

A Fig. \ref{Fig:Modelagem:Curva:passivo:fxd:dy} é o resultante 

\begin{figure}[!htp]
	\centering
	\caption*{Força magnética (N) x Deslocamento em Y (mm)}
	\includegraphics[width=1 \columnwidth,angle=0]{Figs/Simulacoes/Passivo/modelo:circuito:passivo:fXd:calculado:comsol:dy.pdf}
		\caption{Comparativo do cálculo da força magnética pela Eq. \eqref{eq:passivo:Fx} e pelo modelo no Comsol com $\Delta_x = 1.2$ mm}
		\label{Fig:Modelagem:Curva:passivo:fxd:dy}
\end{figure}

\todo[inline]{Verificar o Dy}