\pagestyle{empty}
	\cleardoublepage
\pagestyle{fancy}

\chapter{Considerações Finais} \label{Cap:Consideracoes:Finais}

%\begin{itemize}
%	\item Falar do mancal obtido
%	\item controlador
%	\item outros tipos de materiais (soft iron)
%	\item falar do esforço computacional
%	\item do controle controlar um sistema do matlab no comsol
%\end{itemize}

O mancal magnético obtido com o desenvolvimento dessa dissertação abre portas para a prototipagem de uma roda de reação nacional para controle de atitude de satélites. Com ela é possível vencer as barreiras impostas do uso de um mancal convencional (mecânico) em ambiente espacial.

Em projetos aerospaciais muitas vezes especifica-se atuadores que não possuem equivalentes comerciais, em parte devido ao restrito mercado e em parte pela especificidade do problema. A metodologia aqui proposta pode ser aplicada para a especificação de um mancal com diferentes especificações, sendo facilmente adaptável ao projeto do atuador.

O processo de otimização proposto facilita a escolha dos parâmetros construtivos do mancal já que o problema é reduzido a especificação de desempenho e restrições, só possível devido ao modelo analítico dos campos magnéticos envolvidos.

A utilização de simulações em elementos finitos tornam mais preciso o resultado final, sendo uma ferramenta importante tanto para validação do modelo e para o entendimento do mancal. Porém devido ao longo tempo de processamento e resolução do modelo, não deve ser o único meio de projeto.

A utilização de materiais magnéticos com melhores propriedades magnéticas a que o do utilizado (Aço 1020) pode tornar o mancal resultante com menores dimensões e menos suscetível as correntes induzidas, quando submetido a um campo alternado. 

O modelo dinâmico obtido é de fácil parametrização e depende das forças magnéticas envolvidas (ímã e polos) além da massa do rotor.  

\todo{continuar, falar do controlador}







