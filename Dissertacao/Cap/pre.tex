% Limpa cabeçalhos.
% (solução para lidar com a númeração das páginas pré-textuais).
\pagestyle{empty}

% Capa
\begin{titlepage}

% Se quiser uma figura de fundo na capa ative o pacote wallpaper
% e descomente a linha abaixo.
% \ThisCenterWallPaper{0.8}{nomedafigura}

\begin{center}
{\LARGE \nomedoaluno}
\par
\vspace{200pt}
{\Huge \titulo}
\par
\vfill
\textbf{{\large São Paulo}\\
{\large \the\year}}
\end{center}
\end{titlepage}

% Faz com que a página seguinte sempre seja ímpar (insere pg em branco)
\cleardoublepage

% Numeração em elementos pré-textuais é opcional (ativada por padrão).
% Para desativá-la comente a linha abaixo.
\pagestyle{fancy}

% Números das páginas em algarismos romanos
\pagenumbering{roman}

%% Página de Rosto

% Numeração não deve aparecer na página de rosto.
\thispagestyle{empty}

\begin{center}
{\LARGE \nomedoaluno}
\par
\vspace{200pt}
{\Huge \titulo}
\end{center}
\par
\vspace{90pt}
\hspace*{175pt}\parbox{7.6cm}{{\large  \textbf{Dissertação apresentda à Escola Politécnica da Universidade de São Paulo para o obtenção do título de Mestre em Engenharia de Sistemas}}}
\par
\vspace{1em}
\hspace*{175pt}\parbox{7.6cm}{{\large Orientador: Prof. Dr. José Jaime da Cruz}}

\par
\vfill
\begin{center}
\textbf{{\large São Paulo}\\
{\large \the\year}}
\end{center}

\newpage

% Ficha Catalográfica
\hspace{8em}\fbox{\begin{minipage}{10cm}
Aluno, Nome C.

\hspace{2em}\titulo

\hspace{2em}\pageref{LastPage} páginas

\hspace{2em}Dissertação (Mestrado) - Escola Politécnica da Universidade de São Paulo. Departamento de Controle e Automação.

\begin{enumerate}
\item Mancal Magnético
\item Roda de Reação
\item Controle Multivariável
\end{enumerate}
I. Universidade de São Paulo. Escola Politécnica. Departamento de Eletrônica.

\end{minipage}}
%\par
%\vspace{2em}
\begin{center}
%{\LARGE\textbf{Comissão Julgadora:}}
%
%\par
%\vspace{5em}
%\begin{tabular*}{\textwidth}{@{\extracolsep{\fill}}l l}
%\rule{16em}{1px} 	& \rule{16em}{1px} \\
%Prof. Dr. 		& Prof. Dr. \\
%Nome			& Nome
%\end{tabular*}

\par
\vspace{10em}

\parbox{16em}{\rule{16em}{1px} \\
Prof. Dr. \\
José Jaime da Cruz
}
\end{center}

%\newpage

%% Dedicatória
%% Posição do texto na página
%\vspace*{0.75\textheight}
%\begin{flushright}
%  \emph{Dedicatória...}
%\end{flushright}

\newpage

%% Epígrafe
%\vspace*{0.4\textheight}
%\noindent{\LARGE\textbf{Exemplo de epígrafe}}
%% Tudo que você escreve no verbatim é renderizado literalmente (comandos não são interpretados e os espaços são respeitados)
%\begin{verbatim}
%
%
%\end{verbatim}
%\begin{flushright}
%Lenine e Bráulio Tavares
%\end{flushright}
%
%\newpage

% Agradecimentos

% Espaçamento duplo
\doublespacing

\noindent{\LARGE\textbf{Agradecimentos}}

%Agradeço ao meu orientador pela orientação 

\newpage

\vspace*{10pt}
% Abstract
\begin{center}
  \emph{\begin{large}Resumo\end{large}}\label{resumo}
\vspace{2pt}
\end{center}
% Pode parecer estranho, mas colocar uma frase por linha ajuda a organizar e reescrever o texto quando necessário.
% Além disso, ajuda se você estiver comparando versões diferentes do mesmo texto.
% Para separar parágrafos utilize uma linha em branco.
\noindent
\large{
	Esta dissertação tem como objetivo o projeto de um mancal magnético para rodas de reação com aplicação na malha de controle de atitude de satélites. 
	
	Mancais magnéticos são alternativas aos mancais tradicionais tais como os de esferas ou de lubrificação seco pois trabalham sem contato mecânico entre o rotor e o estator minimizando assim a fricção entre ambas as partes. Além da minimização do atrito, o ganho em confiabilidade e vida útil da roda de reação é considerável por não apresentar desgastes mecânicos.
	
	Devida às consequências de qualquer fricção no movimento relativo entre a inércia (parte rotativa da roda de reação) e o satélite o mancal torna-se um componente crítico da roda de reação. A fricção se traduz não apenas num maior consumo de potência elétrica, como também na introdução de uma zona morta de atuação em torque, bem como na limitação da vida útil da roda de reação devido ao gradual desgaste do mancal. 
		
	O mancal proposto possui dois graus de liberdade axiais ativamente controlados e faz uso de ímãs para a estabilização passiva dos demais graus de liberdade. 
	
	Ao longo do desenvolvimento são apresentados modelos não lineares dos campos magnéticos e das forças atuantes no mancal são encontrados. Com esses modelos, uma otimização é realizada a fim de encontrar melhores características. Um modelo não linear da dinâmica do rotor é desenvolvido e um controle PID capaz de estabilizar o rotor em seu ponto de equilíbrio e com isso, demonstrar a viabilidade da topologia proposta.
}


	 
\par
\vspace{1em}
\noindent\textbf{Palavras-chave:} Mancal Magnético, Rodas de Reação, Modelagem Eletromagnética, Controle Multivariável

\normalsize 

\newpage
%
%% Criei a página do abstract na mão, por isso tem bem mais comandos do que o resumo acima, apesar de serem idênticas.
\vspace*{10pt}
% Abstract
\begin{center}
  \emph{\begin{large}Abstract\end{large}}\label{abstract}
\vspace{2pt}
\end{center}

% Selecionar a linguagem acerta os padrões de hifenação diferentes entre inglês e português.
\selectlanguage{english}
\noindent
\large{

The main objective of this work is to project a magnetic bearing for reaction wheels with application in satellite attitude control.


Magnetic bearings are alternatives to traditional bearings such as ball or dry lubrication because they work without mechanical contact between the rotor and the stator thereby minimizing friction between both parts. In addition to minimizing friction, the gain in reliability and lifetime of the reaction wheel is considerable as a consequence of the absence of wear.

Because of the consequences of any friction in the relative movement between the inertia (of the reaction wheel) and the satellite ( which is rigidly connected to the satellite body), the bearing becomes a critical component of the reaction wheel. The friction gives rise not only to a greater consumption of electric power, as well as the introduction of a torque dead zone operation,  in a reduced lifetime of the reaction wheel due to gradual wear of the bearing.

The proposed bearing has two axial degrees of freedom actively controlled and makes use of magnets for the passive stabilization of other degrees of freedom.

Nonlinear models of magnetic fields and forces acting on the bearing are presented. With these models, an optimization is performed to find the best bearing characteristics. A nonlinear model rotor dynamics is developed and a PID control capable of stabilizing the active degrees of freedom presented.


}

\par
\vspace{1em}
\noindent\textbf{Keywords:} Magnetic Bearing, Reaction Wheel, Multivariable Control, Electromagnetic modeling.

\normalsize 

% Voltando ao português...
\selectlanguage{brazilian}

\newpage

% Desabilitar protrusão para listas e índice
\microtypesetup{protrusion=false}

% Lista de figuras
\listoffigures

% Lista de tabelas
\listoftables

% Abreviações
% Para imprimir as abreviações siga as instruções em 
% http://code.google.com/p/mestre-em-latex/wiki/ListaDeAbreviaturas

% Índice
\tableofcontents

% Re-habilita protrusão novamente
\microtypesetup{protrusion=true}
