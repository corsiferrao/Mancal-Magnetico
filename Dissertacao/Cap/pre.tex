% Limpa cabeçalhos.
% (solução para lidar com a númeração das páginas pré-textuais).
\pagestyle{empty}

%% Capa
%\begin{titlepage}
%
%% Se quiser uma figura de fundo na capa ative o pacote wallpaper
%% e descomente a linha abaixo.
%% \ThisCenterWallPaper{0.8}{nomedafigura}
%
%\begin{center}
%{\LARGE \nomedoaluno}
%\par
%\vspace{200pt}
%{\Huge \titulo}
%\par
%\vfill
%\textbf{{\large São Paulo}\\
%{\large \the\year}}
%\end{center}
%\end{titlepage}

% Faz com que a página seguinte sempre seja ímpar (insere pg em branco)
\cleardoublepage

% Numeração em elementos pré-textuais é opcional (ativada por padrão).
% Para desativá-la comente a linha abaixo.
\pagestyle{fancy}

% Números das páginas em algarismos romanos
\pagenumbering{roman}

%% Página de Rosto

% Numeração não deve aparecer na página de rosto.
\thispagestyle{empty}

\begin{center}
{\LARGE \nomedoaluno}
\par
\vspace{200pt}
{\Huge \titulo}
\end{center}
\par
\vspace{90pt}
\hspace*{175pt}\parbox{7.6cm}{{\large  \textbf{Texto apresentado a Escola Politécnica da Universidade de São Paulo para o Exame de Qualificação de Mestrado em Engenharia de Sistemas}}}%Documento de Qualificação apresentado ao Instituto de Biociências da Universidade de São Paulo, para a obtenção de Título de Mestre em Ciências, na Área de XXXXXXXX.}}

\par
\vspace{1em}
\hspace*{175pt}\parbox{7.6cm}{{\large Orientador: Prof. Dr. José Jaime da Cruz}}

\par
\vfill
\begin{center}
\textbf{{\large São Paulo}\\
{\large \the\year}}
\end{center}

\newpage

%% Ficha Catalográfica
%\hspace{8em}\fbox{\begin{minipage}{10cm}
%Aluno, Nome C.
%
%\hspace{2em}\titulo
%
%\hspace{2em}\pageref{LastPage} páginas
%
%\hspace{2em}Dissertação (Mestrado) - Escola Politécnica da Universidade de São Paulo. Departamento de Controle e Automação.
%
%\begin{enumerate}
%\item Mancal Magnético
%\item Roda de Reação
%\item Controle Multivariável
%\end{enumerate}
%I. Universidade de São Paulo. Escola Politécnica. Departamento de Eletrônica.
%
%\end{minipage}}
%\par
%\vspace{2em}
%\begin{center}
%{\LARGE\textbf{Comissão Julgadora:}}
%
%\par
%\vspace{5em}
%\begin{tabular*}{\textwidth}{@{\extracolsep{\fill}}l l}
%\rule{16em}{1px} 	& \rule{16em}{1px} \\
%Prof. Dr. 		& Prof. Dr. \\
%Nome			& Nome
%\end{tabular*}
%
%\par
%\vspace{10em}
%
%\parbox{16em}{\rule{16em}{1px} \\
%Prof. Dr. \\
%José Jaime da Cruz
%}
%\end{center}
%
%%\newpage
%
%%% Dedicatória
%%% Posição do texto na página
%%\vspace*{0.75\textheight}
%%\begin{flushright}
%%  \emph{Dedicatória...}
%%\end{flushright}
%
%\newpage
%
%%% Epígrafe
%%\vspace*{0.4\textheight}
%%\noindent{\LARGE\textbf{Exemplo de epígrafe}}
%%% Tudo que você escreve no verbatim é renderizado literalmente (comandos não são interpretados e os espaços são respeitados)
%%\begin{verbatim}
%%
%%
%%\end{verbatim}
%%\begin{flushright}
%%Lenine e Bráulio Tavares
%%\end{flushright}
%
%\newpage
%
%% Agradecimentos
%
%% Espaçamento duplo
%\doublespacing
%
%\noindent{\LARGE\textbf{Agradecimentos}}
%
%Agradeço ao meu orientador, ao meu co-orientador, aos meus colaboradores, aos técnicos, à seção administrativa, à fundação que liberou verba para minhas pesquisas, aos meus amigos, à minha família....

\newpage

\vspace*{10pt}
% Abstract
\begin{center}
  \emph{\begin{large}Resumo\end{large}}\label{resumo}
\vspace{2pt}
\end{center}
% Pode parecer estranho, mas colocar uma frase por linha ajuda a organizar e reescrever o texto quando necessário.
% Além disso, ajuda se você estiver comparando versões diferentes do mesmo texto.
% Para separar parágrafos utilize uma linha em branco.
\noindent

	Roda de reação é um sistema excecional para o controle de atitude de satélites, \todo{continuar}
	
	Devido as consequências de qualquer fricção no movimento relativo entre a inércia (parte rotativa) e o satélite (estator) o mancal torna-se um componente crítico da roda de reação. A fricção se traduz não apenas num maior consumo de potência elétrica, como também na introdução de uma zona morta de atuação em torque, bem como na limitação da vida útil da roda de reação devido ao gradual desgaste do mancal. 
	
	Mancais magnéticos são alternativas aos mancais tradicionais (esferas, lubrificação seco) pois trabalham sem contato mecânico entre o rotor e o estator minimizando assim a fricção entre ambas as partes. Além da minimização do atrito, o ganho em confiabilidade e vida útil da roda de reação é considerável por não apresentar desgastes mecânicos.
	 
	Esta dissertação a vem ao encontro de projetar um mancal magnético para rodas de reação com aplicação na malha de controle de atitude de satélites. Propomos uma topologia diferente da encontrada na literatura, com um oito polos e dois graus de liberdade ativo.
	 
	 
\par
\vspace{1em}
\noindent\textbf{Palavras-chave:} Mancal Magnético, Rodas de Reação, Modelagem Eletromagnética
\newpage
%
%% Criei a página do abstract na mão, por isso tem bem mais comandos do que o resumo acima, apesar de serem idênticas.
%\vspace*{10pt}
%% Abstract
%\begin{center}
%  \emph{\begin{large}Abstract\end{large}}\label{abstract}
%\vspace{2pt}
%\end{center}
%
%% Selecionar a linguagem acerta os padrões de hifenação diferentes entre inglês e português.
%\selectlanguage{english}
%\noindent
%		Este trabalho vem ao encontro de projetar um mancal magnético para rodas de reação que tem utilização na malha de controle de atitude de satélites. \ldots
%\par
%\vspace{1em}
%\noindent\textbf{Keywords:} Magnetic Bearing, Reaction Wheel, Multivariable Control

% Voltando ao português...
\selectlanguage{brazilian}

\newpage

% Desabilitar protrusão para listas e índice
\microtypesetup{protrusion=false}

% Lista de figuras
\listoffigures

% Lista de tabelas
\listoftables

% Abreviações
% Para imprimir as abreviações siga as instruções em 
% http://code.google.com/p/mestre-em-latex/wiki/ListaDeAbreviaturas
\printnomenclature

% Índice
\tableofcontents

% Re-habilita protrusão novamente
\microtypesetup{protrusion=true}
