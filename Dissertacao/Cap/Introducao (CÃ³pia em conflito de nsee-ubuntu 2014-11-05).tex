% Faz com que o ínicio do capítulo sempre seja uma página ímpar
\cleardoublepage

% Inclui o cabeçalho definido no meta.tex
\pagestyle{fancy}

% Números das páginas em arábicos
\pagenumbering{arabic}

\chapter{Introdução}\label{intro}

O sistema de controle de atitude e órbita é uma das tecnologias mais críticas de qualquer sistema espacial. O desenvolvimento de um sistema de controle de atitude em território nacional permanece incompleto \citep{Veloso2009} e a venda de componentes deste sistema ao nosso país é frequentemente recusada por países detentores dessa tecnologia.

Basicamente, um sistema de controle de atitude é formado por sensores, atuadores e uma central responsável pelo processamento dos sinais dos sensores e comando dos atuadores, segundo uma lei de controle. Os sensores mais comuns são detectores de horizonte, sensores magnéticos, sensores solares, giroscópios e rastreadores estelares. 

Os principais atuadores incluem: propulsores, torques magnéticos e rodas de reação. A quase totalidade destes componentes de controle possui atualmente alguma iniciativa de desenvolvimento no país, seja por instituições governamentais ou por grupos de pesquisa independentes \citep{PresidenciaRepublica}. A principal exceção são as rodas de reação, que praticamente não têm projetos de desenvolvimento em andamento e, no entanto, representam um componente indispensável na realização de manobras e na estabilização e controle de atitude em três eixos. 

Rodas de reação são dificilmente substituíveis pois apresentam larga faixa de operação em torque (ao contrário de atuadores magnéticos) e são alimentadas pela energia renovável fornecida por painéis solares (ao contrário de propulsores baseados em um estoque finito de combustível). Por estes motivos, rodas de reação estão presentes em praticamente qualquer satélite que apresente requerimentos mínimos de desempenho em atitude.

Uma roda de reação pode ser descrita como um atuador inercial com funcionamento baseado no princípio de conservação do momento angular. A atuação da roda de reação sobre o satélite se realiza por intercâmbio de momento angular, limitado ao eixo de rotação da roda. Devido a grande diferença entre a inercia do satélite e o da roda de reação, um controle de atitude com muita precisão é possível com esse sistema.

Rodas de reação são tipicamente constituídas de um motor elétrico, geralmente um motor sem escovas, um mancal e um elemento de inércia.  O elemento de inércia e o motor são montados sobre o mancal que deve garantir a  precisa rotação em torno de um eixo. A velocidade de rotação do sistema é controlado por uma eletrônica de acionamento do motor. 

% \todo[color=magenta]{Descrever um pouco mais}

\section{Objetivo}

Essa dissertação vem ao encontro de projetar um controle para a malha do mancal magnético da roda de reação que está sendo desenvolvido no Núcleo de Sistemas Eletrônicos Embarcados (NSEE) do Instituto Mauá De Tecnologia (IMT) com apoio do Instituto Nacional De Pesquisas Espaciais (INPE). \todo[color=magenta]{Ref. Roda de Reacão - Mauá}

\section{Justificativa}

A suspensão do rotor com relação ao estator representa uma parte crítica em rodas de reação \cite{taniwaki2003experimental} \todo[color=magenta]{Ref. : Experimental and numerical analysis of reaction wheel disturbances} devido as consequências de qualquer fricção no movimento relativo entre estes dois componentes. Com efeito, a fricção se traduz não apenas em um maior consumo de potência elétrica, como também na introdução de uma zona morta de atuação em torque, bem como na limitação da vida útil da roda de reação devido ao gradual desgaste do mancal.

Uma solução mecânica para a interface entre o rotor e o estator é o mancal por rolamento. Apesar de sua aparente simplicidade, apresenta desafios para a obtenção dos valores mínimos de fricção necessários, em vista das exigências de consumo, controlabilidade e vida útil da roda de reação \citep{Krishnan2010}. No caso de aplicações aerospaciais, a lubrificação do rolamento representa também considerável dificuldade devido à impossibilidade de utilização de lubrificantes tradicionais em condições de baixa ou nenhuma pressão atmosférica, que leva à perda dos componentes voláteis destes lubrificantes e sua consequente degradação. Outra dificuldade se deve à tendência de migração dos lubrificantes na ausência de gravidade, o que costuma ser abordado com estratégias de recaptura ou relubrificação. Sistemas de relubrificação, em particular, apresentam grande complexidade e seu comportamento orbital é de difícil validação em laboratório.

A outra solução proposta é a utilização de um mancal magnético \citep{Bangcheng2012},  que é uma alternativa sem contato mecânico entre o rotor e o estator, na qual o rotor é mantido suspenso magneticamente. O ganho em confiabilidade e vida útil da roda de reação é considerável \citep{Marble2006}, sendo a vida útil basicamente limitada  pela durabilidade da eletrônica. A operação sem contato elimina a necessidade de lubrificante e possibilita consequentemente a operação em vácuo, o que se traduz em simplificação nos requisitos da concepção mecânica. 
A ausência de fricção elimina a zona morta de aplicação de torque em baixas velocidades, eliminando não-linearidades da lei de controle\todo[color=magenta]{Ref.}, 
além de possibilitar a eliminação de defeitos de balanceamento e vibrações mecânicas, com consequente ganho em simplicidade dos algoritmos e em desempenho do controle de atitude. 
A contrapartida é a adição de uma malha de controle para a suspensão eletromagnética. O ganho de eficiência trazido pela ausência de fricção também é contrabalanceado, ao menos parcialmente, pelo consumo de potência dos atuadores deste tipo de mancal.


%O controle de mancal magnético é de grande interesse para a área de controle com aplicações em diferentes áreas

\section{Revisão bibliográfica}

\begin{footnotesize}
\textbf{Conteúdo:}
\begin{enumerate}
	\item Historia do Mancal Magnético
	\item Falar de que não pode existir uma mancal puramente passivo (Earnshaw's theorem)
	\item Tipos de MM (graus de liberdade)
	\item Utilização do MM.
	\item Detalhar uso em satélites (exaltar problemas)
	\item Técnicas de controle aplicadas em MM.
\end{enumerate}
\end{footnotesize}

Estudo iniciais sobre mancais magnéticos são datados do seculo XIX \citep{Weise1989} porem só recentemente começaram a ser aplicados na industria \citep{}



\section{Descrição do conteúdo}
\todo[inline]{Descrição do documento em geral}