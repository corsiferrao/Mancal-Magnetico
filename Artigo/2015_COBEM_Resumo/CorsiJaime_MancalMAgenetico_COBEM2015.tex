% LaTeX .tex example for the proceedings of
% COBEM 2015 - 23rd International Congress of Mechanical Engineering
% November, 6-11 2015 - Rio de Janeiro, RJ, Brazil
%
% Based on the template of the proceedings of COBEM2013 

\documentclass[10pt,fleqn,a4paper,twoside]{article}
\usepackage{cobem2015}
\def\shortauthor{R. Corsi, J. Jaime}
\def\shorttitle{Active Magnetic Bearing Project For a Satellite Reaction Wheel}


\begin{document}
\fphead
\hspace*{-2.5mm}\begin{tabular}{||p{\textwidth}}
\begin{center}
\vspace{-4mm}
\title{Active Magnetic Bearing Project For a Satellite Reaction Wheel}
\end{center}
\authors{Rafael Corsi Ferr\~{a}o} \\
\authors{Jos\'{e} Jaime da Cruz} \\
\institution{Escola Polit\'{e}cnica da Universidade de
S\~{a}o Paulo} \\
\institution{rafael.corsi@usp.br, jaime@lac.usp.br} \\

\abstract{\textbf{Abstract.} 

In this paper, the development of a novel active magnetic bearing (MB) system for reaction wheels applicable in satellite attitude control is presented. The proposed bearing has four degrees of freedom passively stable (EMB) by one pair of permanent magnet; two degrees of freedom (AMB) are actively stabilized by eight electromagnetic poles. The  magnetic model of both EMB and AMB are presented and  equations of force-current and force-position are analyzed by the magnetic circuit approach and by the finite element method. With the force characteristic curves a non-linear dynamic model for the MB and a control system that stabilizes the bearing at its operating point are presented. A flat, uncoupled and scalable magnetic bearing with good stiffness, that can be used on satellites reaction wheels to improve its performance and reliability, is obtained. A prototype is under construction. Simulation results are presented.

}\\
\\
\keywords{\textbf{Keywords:} Active Magnetic Bearing, Reaction Wheel, Attitude Control}\\
\end{tabular}


\end{document}
